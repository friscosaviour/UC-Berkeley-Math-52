\documentclass[12pt]{article}
\usepackage{amsmath, amssymb}
\usepackage{geometry}
\geometry{margin=1in}

\title{Integration Rules and Properties with Explanations}
\author{Calculus II Notes}
\date{}

\begin{document}

\maketitle

\section*{Introduction}
Integration is the reverse process of differentiation. While derivatives measure \textit{instantaneous change}, integrals measure \textit{accumulation}. 
For example, if velocity is the derivative of position, then the integral of velocity gives you the total change in position. 
These notes cover the most important properties and formulas, along with explanations of why they make sense. 

\section*{1. Basic Properties of Integrals}

\subsection*{Linearity}
\[
\int (f(x) + g(x)) \, dx = \int f(x) \, dx + \int g(x)\, dx
\]
\[
\int c \cdot f(x)\, dx = c \int f(x)\, dx
\]

\textbf{Explanation:} Integration is additive and respects scaling, just like summation. Since integrals measure ``area under a curve,'' the area under $f+g$ is just the sum of their areas, and multiplying a function by a constant $c$ multiplies the area by $c$. 

\subsection*{Integral of a Constant}
\[
\int k \, dx = kx + C
\]

\textbf{Explanation:} A constant function is just a horizontal line. Its integral is the area of a rectangle with height $k$ and base $x$. That’s why the antiderivative is $kx$. 

\section*{2. Power Rule for Integration}

\[
\int x^n \, dx = 
\begin{cases}
\frac{x^{n+1}}{n+1} + C, & n \neq -1 \\
\ln|x| + C, & n = -1
\end{cases}
\]

\textbf{Explanation:} This is the ``opposite'' of the power rule for derivatives. Since $\frac{d}{dx}\left( \frac{x^{n+1}}{n+1}\right) = x^n$, the integral must undo that step. The $n=-1$ case is special: $\frac{1}{x}$ integrates to $\ln|x|$ because the derivative of $\ln|x|$ is $1/x$. 

\section*{3. Exponential and Logarithmic Functions}

\[
\int e^x \, dx = e^x + C
\]
\[
\int a^x \, dx = \frac{a^x}{\ln a} + C, \quad (a>0, a \neq 1)
\]
\[
\int \frac{1}{x} \, dx = \ln|x| + C
\]

\textbf{Explanation:} Exponentials are special: $e^x$ is its own derivative and also its own integral. For bases $a^x$, the factor of $\frac{1}{\ln a}$ comes from the chain rule in reverse. The logarithm shows up naturally because it is the inverse of the exponential. 

\section*{4. Trigonometric Functions}

\begin{align*}
\int \sin x \, dx &= -\cos x + C \\
\int \cos x \, dx &= \sin x + C \\
\int \sec^2 x \, dx &= \tan x + C \\
\int \csc^2 x \, dx &= -\cot x + C \\
\int \sec x \tan x \, dx &= \sec x + C \\
\int \csc x \cot x \, dx &= -\csc x + C
\end{align*}

\textbf{Explanation:} These follow directly from knowing derivatives of sine, cosine, and tangent. For example, since $\frac{d}{dx}(\cos x) = -\sin x$, the integral of $\sin x$ must be $-\cos x$. Think of integration as ``asking which function has this as a derivative.'' 

\section*{5. Inverse Trigonometric Functions}

\begin{align*}
\int \frac{1}{\sqrt{1 - x^2}} \, dx &= \arcsin x + C \\
\int \frac{1}{1 + x^2} \, dx &= \arctan x + C \\
\int \frac{1}{|x|\sqrt{x^2 - 1}} \, dx &= \arcsec x + C
\end{align*}

\textbf{Explanation:} These come from memorizing the derivatives of inverse trig functions and working backwards. For instance, since $\frac{d}{dx}(\arctan x) = \frac{1}{1+x^2}$, the integral of $\frac{1}{1+x^2}$ must be $\arctan x$. They often show up when you do trigonometric substitution. 

\section*{6. Techniques of Integration}

\subsection*{a. Substitution Rule}
If $u = g(x)$ and $g'(x)$ is continuous:
\[
\int f(g(x)) g'(x) \, dx = \int f(u) \, du
\]

\textbf{Explanation:} This is the ``reverse chain rule.'' If differentiation requires the chain rule, integration requires undoing it. When a function is nested, substitution helps by renaming the inside function $u$. 

\subsection*{b. Integration by Parts}
\[
\int u \, dv = uv - \int v \, du
\]

\textbf{Explanation:} This is the ``reverse product rule.'' It’s useful when the integrand is a product of two functions where one becomes simpler when differentiated, and the other doesn’t become worse when integrated. A common memory trick: \textbf{LIATE} (Log, Inverse trig, Algebraic, Trig, Exponential) helps pick $u$. 

\subsection*{c. Symmetry Rules}
\[
\int_{-a}^{a} f(x)\, dx = 0 \quad \text{if $f$ is odd}
\]
\[
\int_{-a}^{a} f(x)\, dx = 2\int_{0}^{a} f(x)\, dx \quad \text{if $f$ is even}
\]

\textbf{Explanation:} Odd functions have symmetric positive and negative areas that cancel out. Even functions are mirror images, so you can double the area from $0$ to $a$. Recognizing symmetry saves computation time. 

\end{document}
